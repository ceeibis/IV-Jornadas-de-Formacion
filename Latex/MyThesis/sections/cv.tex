\noappendix{\textit{Curriculum Vitae}}%\label{sec:}

\newcommand{\cab}[1]{\hspace{-0cm}\textcolor{blue}{\rule{3mm}{3mm}{\Huge #1}}}
\newcommand{\subcab}[1]{\textcolor{blue}{{\Large #1}}}

%\begin{center}
%\begin{tabular}{lcr}
%\hspace{-1cm}\bf\it\huge \textcolor{skyblue}{CURRICULUM VITAE} 
%&
%\hspace{2.5cm}
%&
%{\textcolor{skyblue}{\bf\Large Enrique Blanco Garc\'{\i}a}}
%\end{tabular}
%\textcolor{verydarkblue}{Last update: \today}
%\end{center}

%\vspace{-2cm}
%\begin{center}
%\incgraph{width=0.35\linewidth}{ps/WallaceandMe}
%\end{center}
%\vspace{-1cm}
%\vspace{2cm}

%\vspace{-1cm}
%\noindent\cab{  A. PERSONAL DATA}\\
\sectionred*{PERSONAL DATA}

\begin{tabular}{ll}
\textbf{Name}: &  Enrique Blanco Garc\'{\i}a\\
\textbf{Birthplace and birthdate}: &  Barcelona, January 12th. 1976\\
\textbf{Working Address}: &  Centre de Regulaci\'o Gen\`omica\\
 & Passeig de la Barceloneta 37-49\\
 & Barcelona\\
\textbf{Telephone number}: &  +34 93 224 08 91\\
\textbf{E-mail}: & eblanco@imim.es\\
\textbf{Web page:} & http://genome.imim.es/$\sim$eblanco\\[2ex]
\end{tabular}

%\vspace{0.5cm}
%\noindent\cab{   B. ACADEMIC CURRICULUM}
\sectionred*{ACADEMIC CURRICULUM}

\begin{itemize}
\item
\ver{Engineer in Computer Science} (\emph{Ingeniero superior en Inform\'atica}). Facultat 
d\`{ }inform\`atica de Barcelona. Universitat Polit\`ecnica de Catalunya, Spain (June 2000).
[Mark: 7.40/10, PFC: MH]
\item 
\ver{DEA in Algorithmics} (\emph{Diploma de Estudios Avanzados, Research Sufficiency}). 
Departament de Llenguatges i Sistemes Informatics. Facultat d\`{ }inform\`atica de Barcelona. 
Universitat Polit\`ecnica de Catalunya , Spain (June 2002).
\item
\ver{AQU certificate}: Professorat Col.laborador (teaching staff), 25 November 2005.
\end{itemize}

\vspace{0.5cm}
\subcab{Language Skills}

\begin{itemize}
\item English : \ver{Advanced level (Certificat d'\ Aptitud) (Level C)}, Official School of Languages, 
Barcelona (EOIBD), Spain.
\item Italian : \ver{Elementary level (Certificat Elemental) (Level B)}, Official School of Languages, 
Barcelona (EOIBD), Spain.
\item Catalan and Spanish : mother tongues.
\end{itemize}

%\vspace{0.5cm}
%\noindent\cab{   C. RESEARCH CURRICULUM}
\sectionred*{RESEARCH CURRICULUM}

\begin{itemize}
\item 2001 - 2006. PhD student (Software program, Universitat Polit\`ecnica de Catalunya) 
at Genome Informatics Research Lab, IMIM, Barcelona.\\ 
\noindent PhD supervisors:
\begin{itemize}
\item Dr. Xavier Messeguer - peypoch@lsi.upc.edu\\
(Facultat d\`{ }inform\`atica de Barcelona. Universitat Polit\`ecnica de Catalunya) 
\item
Dr. Roderic Guig\'{o} - rguigo@imim.es\\
(Genome Informatics Research Lab, Research Group of Medical Informatics. 
IMIM-UPF-CRG).
\end{itemize}

\item 1999 - 2000. Programmer in Genome Informatics Research Lab, Research Group 
of Medical Informatics, at IMIM, Barcelona.
\end{itemize}

\vspace{0.5cm}
\subcab{Research areas}
\begin{enumerate}
\item Bioinformatics (algorithmics)
\begin{itemize}
\item Sequence analysis
\item Sequence and map alignments
\item Multiple alignments
\item Representation of biological signals
\end{itemize}

\item Bioinformatics (computational biology)
\begin{itemize}
\item Characterization of gene regulatory regions
\item Gene expression
\item Comparative genomics
\item Microarray analysis
\item Computational gene prediction
\end{itemize}

\item Computer Science
\begin{itemize}
\item Algorithmics
\item Artificial intelligence
\item Parallelism and supercomputation
\item Internet aplications
\end{itemize}
\end{enumerate}

\vspace{0.5cm}
\subcab{Computer Skills}
\begin{itemize}
\item
Programming languages: Perl, C, C++, Java, LISP, Pascal, Modula, Ada, PVM, Prolog, GAWK
\item
Document edition: \LaTeX, \texttt{pdflatex}
\item
Web design: XML, HTML, JavaScript, CGI-scripts (web servers), Macromedia Flash, CSSs
\item
Operating systems:  Linux, MAC OS X, Irix, Solaris, Windows 95/98/00/XP
\item
Office: Word, PowerPoint, Excel, Access
\end{itemize}



\vspace{1cm}
\subcab{Publications}

\begin{itemize}
\item
\textbf{E. Blanco}, X. Messeguer, T.F. Smith and R. Guig\'{o}. Transcription Factor Map Alignment of Promoter Regions. \emph{PLOS Computational Biology}, 2(5):e49(2006). 
\item
\textbf{E. Blanco}, D. Farre, M. Alb\`a, X. Messeguer, and R. Guig\'{o}. ABS: a database of Annotated regulatory Binding Sites from orthologous promoters. \emph{Nucleic Acids Research}, 34:D63-D67 (2006). 
\item
\textbf{E. Blanco} and R. Guig\'{o}. Predictive Methods Using DNA Sequences. In A. D. Baxevanis and B. F. Francis Ouellette, chief editors: \emph{Bioinformatics: A Practical Guide to the Analysis of Genes and Proteins, Third Edition}. John Wiley \& Sons Inc., New York (2005).
ISBN: 0-471-47878-4. 

\item
S. Castellano, S.V. Novoselov, G.V. Kryukov, A. Lescure, \textbf{E. Blanco}, A. Krol. V.N. Gladyshev and R. Guig\'{o}. Reconsidering the evolution of eukaryotic selenoproteins: a novel non-mammalian family with scattered phylogenetic distribution. \emph{EMBO reports}, 5(1):71-77 (2004). 

\item 
S. Beltran, \textbf{E. Blanco}, F. Serras, B. Perez-Villamil, R. Guig\'{o}, S. Artavanis-Tsakonas and 
M. Corominas. Microarray analysis of the transcriptional network controlled by the trithorax group 
gene ash2 in Drosophila melanogaster, \emph{PNAS}, 100: 3293-3298, (2003). 

\item
\textbf{E. Blanco}, G. Parra and R. Guig\'{o}. Using geneid to Identify Genes. In A. Baxevanis and D.B. Davidson, chief editors: \emph{Current Protocols in Bioinformatics}. Volume 1, Unit 4.3 (1-26). John Wiley \& Sons Inc., New York, (2002). ISBN: 0-471-25093-7.

\item 
G. Parra, \textbf{E. Blanco}, and R. Guig\'{o}. geneid in Drosophila. \emph
{Genome Research}, 10: 511-515, (2000).
\end{itemize}

\vspace{0.5cm}
\subcab{Posters}
\begin{itemize}
\item
\textbf{E. Blanco}, M. Pignatelli, X. Messeguer and R. Guig\'{o}. 
``Deconstructing the position weight matrices to detect regulatory elements. Systems Biology meeting: global regulation of gene expression''. \emph{Cold Spring Harbor: global regulation of gene expression}. (March 2005, New York, USA). 

\item
\textbf{E. Blanco}, X. Messeguer and R. Guig\'{o}. 
``Novel computational methods to chracterize regulatory regions. Systems Biology meeting: genomic approaches to transcriptional regulation''. \emph{Cold Spring Harbor: genomic approaches to transcriptional regulation}. (March 2004, New York, USA). 

\item
\textbf{E. Blanco}, X. Messeguer and R. Guig\'{o}. 
``Alignment of Promoter Regions by Mapping Nucleotide Sequences into Arrays of Transcription 
Factor Binding Motifs''. \emph{Seventh annual internation conference on computational biology-RECOMB}. (April 2003, Berlin, Germany).

\item
\textbf{E. Blanco}, G. Parra, S. Castellano, J.F. Abril, M. Burset, X. Fustero, X. Messeguer 
and R. Guig\'{o}. ``Gene prediction in the post-genomic era''. \emph{9-th international conference on Intelligent Systems in Molecular Biology}. (July 2001, Copenhaguen, Denmark).

\item J.F. Abril, \textbf{E. Blanco}, M. Burset, S. Castellano, X. Fustero,  G. Parra and  R. Guig\'o; ``Genome Informatics Research Laboratory: Main Research Topics.''{\it I Jornadas de 
Bioinform\'atica} (June 2000, Cartagena, Spain). 
\end{itemize}

\vspace{0.5cm}
\subcab{Grants}
\begin{itemize}
\item
Predoctoral fellowship. Formacion de Personal Investigador (FPI). Ministerio de Educacion y 
Ciencia (Spain), 2001-2004.
\item
Predoctoral fellowship. Institut Municipal d'Investigacio Medica (Spain), 2005-2006.
\end{itemize}

\vspace{0.5cm}
\subcab{Participation in Research Projects}
\begin{itemize}
\item
Plan Nacional I+D (2003-2006), ref. BIO2003-05073, Ministerio de Ciencia
y Tecnologia (Spain). Principal investigator: Dr. R. Guig\'o i Serra.
\item
Plan Nacional I+D (2000-2003), ref. BIO2000-1358-C02-02 Ministerio de Ciencia
y Tecnologia (Spain). Principal investigator: Dr. R. Guig\'o i Serra.
\end{itemize}

%\vspace{0.5cm}
%\noindent\cab{   D. TEACHING CURRICULUM}
\sectionred*{TEACHING CURRICULUM}

\vspace{0.5cm}
\subcab{Topics}

\begin{itemize}
\item Sequence alignment
\item Dynamic programming
\item Data structures
\item Bioinformatics
\item Weight matrices
\item Likelihood ratios
\item Pattern discovery (EM)
\item Computational gene prediction
\item Promoter characterization
\item Genome browsers on internet
\item Artificial neural nets
\item Markov models
\item Hidden Markov models
\item The Human Genome Project
\item DNA computing
\item Introduction to UNIX
\end{itemize}

\vspace{0.5cm}
\subcab{Teaching Activities}\\

\textcolor{blue}{\rule{3mm}{3mm}{\Large $2006$}}

\begin{itemize}
\item Participation in the master \emph{Tecnologie bioinformatiche applicate alla medicina personalizzata} (Genefinding: 
a primer). Consorzio21/Polaris - parco scientifico e tecnologico della Sardegna. Pula (Italy). [Master, 20h]

\item January-March. Participation in the course \emph{Bioinformatica} at Facultat de Ciencies
de la Salut i de la Vida. Universitat Pompeu Fabra. Barcelona (Spain). [University degree, 60h]\\
\end{itemize}

\textcolor{blue}{\rule{3mm}{3mm}{\Large $2005$}}
\begin{itemize}
\item Participation in the course \emph{Bioinformatica} at Facultat de Ciencies
de la Salut i de la Vida. Universitat Pompeu Fabra. Barcelona (Spain). [University degree, 60h]

\item Participation in the Phd course \emph{Eines informatiques per a genetica molecular} 
(Computational Gene Prediction). PhD program in Genetics. Facultat de Biologia. Universitat de 
Barcelona. Barcelona (Spain). [PhD program, 5h]

\item Participation in the summer course \emph{Bioinformatica per a tothom} (Genome analysis). 
Universitat d'Estiu de la Universitat Rovira i Virgili. Reus (Spain). [Summer course, 10h] 

\item Participation in the summer course \emph{Bioinformatica} (Computational Gene Prediction). 
Universidad Complutense de Madrid. Madrid (Spain). [Summer course, 6h] 

\item Participation in the master \emph{Bioinformatics for health sciences} (Introduction to 
the UNIX environment). Universitat Pompeu Fabra. Barcelona (Spain). [Master, 10h]\\
\end{itemize}

\textcolor{blue}{\rule{3mm}{3mm}{\Large $2004$}}
\begin{itemize}
\item Participation in the course \emph{Bioinformatica} at Facultat de Ciencies
de la Salut i de la Vida. Universitat Pompeu Fabra. Barcelona (Spain). [University degree, 60h]

\item Participation in the Phd course \emph{Eines informatiques per a genetica molecular} 
(Computational Gene Prediction). PhD program in Genetics. Facultat de Biologia. Universitat de 
Barcelona. Barcelona (Spain). [PhD program, 5h]

\item Participation in the summer course \emph{Bioinformatica} (Computational Gene Prediction). 
Universidad Complutense de Madrid. Madrid (Spain). [Summer course, 5h]

\item Participation in the master \emph{Bioinformatics for health sciences} (Introduction to 
the UNIX environment). Universitat Pompeu Fabra. Barcelona (Spain). [Master, 10h] 

\item Participation in the workshop on \emph{Computational genome analysis} at Cosmocaixa,
Fundaci\'o La Caixa. Barcelona (Spain). [Workshop, 4h] 

\item Participation in the \emph{Postgraduate programme in Bioinformatics} (Computational Gene 
Prediction). Universidade de Lisboa / Gulbenkian Institute. Lisbon (Portugal). [Master, 40h]\\ 
\end{itemize}

\textcolor{blue}{\rule{3mm}{3mm}{\Large $2003$}}
\begin{itemize}

\item Participation in the course \emph{Bioinformatica} at Facultat de Ciencies
de la Salut i de la Vida. Universitat Pompeu Fabra. Barcelona (Spain). [University degree, 60h]

\item Participation in the Phd course \emph{Eines informatiques per a genetica molecular} 
(Computational Gene Prediction). PhD program in Genetics. Facultat de Biologia. Universitat de 
Barcelona. Barcelona (Spain). [PhD program, 5h]

\item Participation in the master \emph{Bioinformatica y biologia computacional} 
(Computational Gene Prediction). Universidad Complutense de Madrid. Madrid (Spain). [Master, 4h]\\
\end{itemize}

\textcolor{blue}{\rule{3mm}{3mm}{\Large $2002$}}
\begin{itemize}
\item Participation in the course \emph{Bioinformatica} at Facultat de Ciencies
de la Salut i de la Vida. Universitat Pompeu Fabra. Barcelona (Spain). [University degree, 60h]
\item Participation in the course \emph{Bioinformatica} (Genome analysis) at ALMA bioinformatics.
Madrid (Spain). [Course, 8h]\\
\end{itemize}

\textcolor{blue}{\rule{3mm}{3mm}{\Large $2001$}}
\begin{itemize}
\item Participation in the EMBL course \emph{Bioinformatics for comparative and functional 
genomics} (Computational analysis of promoter regions). Universitat Pompeu Fabra. 
Barcelona (Spain). [Course, 2h]\\
\end{itemize}

\textcolor{blue}{\rule{3mm}{3mm}{\Large $2000$}}
\begin{itemize}
\item Participation in the EMB-net course \emph{Bioinformatics} (Computational gene 
identification). Gulbenkian Institute. Lisbon (Portugal). [Course, 20h]\\
\end{itemize}

\vspace{0.5cm}
\subcab{Attended conferences}\\

\begin{itemize}

\item Cold Spring Harbor Labs: global regulation of gene expression. (March 2005, New York, USA). 

\item Cold Spring Harbor Labs: genomic approaches to transcriptional regulation. (March 2004, New York, USA). 

\item IV Jornadas de Bioinform\'atica Espa\~nolas (September 2003, A Coru\~na, Spain). 

\item Seventh annual internation conference on computational biology-RECOMB. (April 2003, Berlin, Germany).

\item Workshop sobre bioinformatica y biologia computacional. Fundacion BBVA. (April 2002, Madrid, Spain).

\item 9-th international conference on Intelligent Systems in Molecular Biology. (July 2001, Copenhaguen, Denmark).

\item I Jornadas de Bioinform\'atica Espa\~nolas (June 2000, Cartagena, Spain). 

\item Jornada Catalana de Supercomputaci\'on. Parque tecnol\'ogico de la Universidad de 
Barcelona (October 1999, Barcelona). 

\item Segunda jornada cient\'ifica sobre an\'alisis computacional de biomol\'eculas. IMIM-UPF (October 1999, Barcelona).
\end{itemize}





