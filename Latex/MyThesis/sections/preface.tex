\chapter*{\textcolor{blue}{\textbf{P}reface}}
\addstarredchapter{Preface}

\lettrine[lines=4,loversize=-0.1,lraise=0.1,lhang=.2]{A}{s a famous dirty detective once said}, 
there must be a hundred good reasons why I shouldn't have just initiated a PhD thesis. But 
right now, I can't think of a single one. On the contrary, I wonder who would have rejected 
the appealing proposal to investigate the genomic world, which is actually the center of the life, 
designing programs on a high-performance computational environment. 

The construction of the first modern computers was one of the major landmarks achieved by 
the human being in the past century. Since then, the application of computers on many 
intriguing problems and the constant evolution of the programs that govern them have permitted 
the researchers in many areas to discover new concepts that would have been otherwise 
unreachable for our generation without this technology.

Molecular biology is not an exception. The sequencing of the human genome would be still 
an impossible challenge if many automatic procedures that are now familiar to us would 
have not been developed before. In this context, Bioinformatics has been the relevant driving 
force responsible for stimulating the advance in the study of the biology of our cells.
Particularly, many clues to understand the life in our planet can be found in the regulation 
of gene expression. Nonetheless, to be sincere I have to admit that we are still completely 
ignorant: a huge amount of new biological information is constantly released so that the global 
picture that we want to reconstruct becomes today somehow even more complex than the day before.

Understanding life is an enormous challenge. In other scale, a PhD is also an exciting challenge 
for a student. It is a period in which not only such a person acquires a valuable education 
in many aspects of his life. At the same time, this individual is supposed to be capable
of applying such knowledge in the investigation of a real problem, sometimes in competition 
with other people that have much more experience. In my case, the task became even more complex
as a computer scientist needs a solid biological background to approach this kind of problems.

This thesis not only pretends to communicate the different phases of my work during the PhD
period of research. Before starting to write, it was also my commitment to elaborate a
manuscript fulfilling the highest requirements of quality and accuracy in the 
the material that is presented. This manuscript attempts to follow a logical and continuous 
argument from the introductory parts to the specific chapters devoted to the presentation 
of the results of the thesis. In addition, a DVD with supplementary materials such as 
the electronic thesis, the bibliography, the software or several educational resources, 
is also released as an excellent complement to the thesis.

The experience and the abilities I have personally acquired during this period do not fit in 
just two hundred pages. From my point of view, the most relevant result of a thesis is not
the compilation of scientific papers published during that time (these should be seen as a 
relevant consequence of a good work). On the contrary, I am totally convinced that the essential 
result of a PhD thesis is the improvement of the individual that positively changes his life 
in many aspects, producing an amazing enrichment of his personality.

In our childhood, many of us have got an intimate and naive desire of changing the 
world to improve it. Surprisingly after so many years, I still have this feeling although 
I am quite conscious that some things are not so easy to be changed whereas others simply 
can not be changed. However, I am happy to see that I have acquired a solid education that 
will be very useful to face more complicate situations throughout my life. In fact, this PhD thesis 
has not represented for me a central objective but an excellent opportunity to stop and learn, 
driving me to more ambitious challenges.

The education of our society has been always among my priorities. To be able to teach is necessary 
to learn to teach before. This is reflected in the fact that I have voluntarily performed hundreds 
of teaching activities during my thesis, always with a high degree of motivation in my presentations. 
Throughout our lives we do not cease to gain new knowledge. But we investigators have the duty of 
communicating rigorously our achievements with honesty to our people at schools, institutes, universities, 
meetings and mass media. To reach this ambitious objective is necessary to be engaged and involved 
in such a project. If we fail now in this attempt, I suspect that the gap between those that have 
the power to learn and investigate and those that do not, will be dangerously large, probably too much. 


\begin{flushright}
%\textit{\thyauthor}\\
\incgraph{width=0.25\linewidth}{ps/signature_eblanco}\\
Barcelona, \thydate\\
\end{flushright}



